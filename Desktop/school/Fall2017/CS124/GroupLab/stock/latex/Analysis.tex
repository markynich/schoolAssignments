This program features around five different stocks with the user's option of auctioning for shares, bidding for buys, and asking for a price. Each stock will have a graph and show how much each stock is worth. When the program runs, it will present the user the market of five different stocks. Each stock has a unique value and can change as the user shares, buys, and asks. Each of the three functions have a button for the user to click to execute the function. Each time the user executes buy function, it will enter the value in a heap tree if it's not already in there. Also the user can buy that amount of shares for the price they are buying/bidding for. The shares function allows the user to enter how many shares he or she wants to buy or sell. Then the ask function lets the user enter the amount she or he wants for selling the amount of shares they are listing. The output of the function will display how many shares he or she have bought or sell for that stock. The user will then see how much they bought each share for and the commission they will be getting. Finally, it will print out the total worth of shares the user bought. Then it will display the lastest bid, highest bid, and lowest bid. Also, the user is allowed to ask the market to display the lastest bid, highest bid, and lowest bid or a specific stock. It will also display the bid size and ask size of how many shares they bought or sell. When the user adds the highest bid, it will move the heap tree around for the highest bid to be added onto the tree and replace the print out of highest bid when the user asks for details of the stock. The same applies for lowest bid. The heap tree will move around the values until the value has a spot and if the value was the lowest, it will display that value the next time the user asks for details of the stock i.\+e. highest,lowest, latest, shares amount, etc. 